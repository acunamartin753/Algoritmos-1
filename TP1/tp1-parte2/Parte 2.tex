\documentclass[a4paper]{article} 

\setlength{\parskip}{0.1em}
\newcommand{\tab}{~ \qquad}
\usepackage{ifthen}
\usepackage{amssymb}
\usepackage{multicol}
\usepackage{graphicx}
\usepackage[absolute]{textpos}
\usepackage{amsmath, amscd, amssymb, amsthm, latexsym}
\usepackage[noload]{qtree}
%\usepackage{xspace,rotating,calligra,dsfont,ifthen}
\usepackage{xspace,rotating,dsfont,ifthen}
\usepackage[spanish,activeacute]{babel}
\usepackage[utf8]{inputenc}
\usepackage{pgfpages}
\usepackage{pgf,pgfarrows,pgfnodes,pgfautomata,pgfheaps,xspace,dsfont}
\usepackage{listings}
\usepackage{multicol}
\usepackage{todonotes}
\usepackage{url}
\usepackage{float}
\usepackage{framed}


\makeatletter

\@ifclassloaded{beamer}{%
  \newcommand{\tocarEspacios}{%
    \addtolength{\leftskip}{4em}%
    \addtolength{\parindent}{-3em}%
  }%
}
{%
  \usepackage[top=1cm,bottom=2cm,left=1cm,right=1cm]{geometry}%
  \usepackage{color}%
  \newcommand{\tocarEspacios}{%
    \addtolength{\leftskip}{5em}%
    \addtolength{\parindent}{-3em}%
  }%
}

\newcommand{\encabezadoDeProc}[4]{%
  % Ponemos la palabrita problema en tt
%  \noindent%
  {\normalfont\bfseries\ttfamily proc}%
  % Ponemos el nombre del problema
  \ %
  {\normalfont\ttfamily #2}%
  \
  % Ponemos los parametros
  (#3)%
  \ifthenelse{\equal{#4}{}}{}{%
  \ =\ %
  % Ponemos el nombre del resultado
  {\normalfont\ttfamily #1}%
  % Por ultimo, va el tipo del resultado
  \ : #4}
}

\newcommand{\encabezadoDeTipo}[2]{%
  % Ponemos la palabrita tipo en tt
  {\normalfont\bfseries\ttfamily tipo}%
  % Ponemos el nombre del tipo
  \ %
  {\normalfont\ttfamily #2}%
  \ifthenelse{\equal{#1}{}}{}{$\langle$#1$\rangle$}
}

% Primero definiciones de cosas al estilo title, author, date

\def\materia#1{\gdef\@materia{#1}}
\def\@materia{No especifi\'o la materia}
\def\lamateria{\@materia}

\def\cuatrimestre#1{\gdef\@cuatrimestre{#1}}
\def\@cuatrimestre{No especifi\'o el cuatrimestre}
\def\elcuatrimestre{\@cuatrimestre}

\def\anio#1{\gdef\@anio{#1}}
\def\@anio{No especifi\'o el anio}
\def\elanio{\@anio}

\def\fecha#1{\gdef\@fecha{#1}}
\def\@fecha{\today}
\def\lafecha{\@fecha}

\def\nombre#1{\gdef\@nombre{#1}}
\def\@nombre{No especific'o el nombre}
\def\elnombre{\@nombre}

\def\practicas#1{\gdef\@practica{#1}}
\def\@practica{No especifi\'o el n\'umero de pr\'actica}
\def\lapractica{\@practica}


% Esta macro convierte el numero de cuatrimestre a palabras
\newcommand{\cuatrimestreLindo}{
  \ifthenelse{\equal{\elcuatrimestre}{1}}
  {Primer cuatrimestre}
  {\ifthenelse{\equal{\elcuatrimestre}{2}}
  {Segundo cuatrimestre}
  {Verano}}
}


\newcommand{\depto}{{UBA -- Facultad de Ciencias Exactas y Naturales --
      Departamento de Computaci\'on}}

\newcommand{\titulopractica}{
  \centerline{\depto}
  \vspace{1ex}
  \centerline{{\Large\lamateria}}
  \vspace{0.5ex}
  \centerline{\cuatrimestreLindo de \elanio}
  \vspace{2ex}
  \centerline{{\huge Pr\'actica \lapractica -- \elnombre}}
  \vspace{5ex}
  \arreglarincisos
  \newcounter{ejercicio}
  \newenvironment{ejercicio}{\stepcounter{ejercicio}\textbf{Ejercicio
      \theejercicio}%
    \renewcommand\@currentlabel{\theejercicio}%
  }{\vspace{0.2cm}}
}


\newcommand{\titulotp}{
  \centerline{\depto}
  \vspace{1ex}
  \centerline{{\Large\lamateria}}
  \vspace{0.5ex}
  \centerline{\cuatrimestreLindo de \elanio}
  \vspace{0.5ex}
  \centerline{\lafecha}
  \vspace{2ex}
  \centerline{{\huge\elnombre}}
  \vspace{5ex}
}


%practicas
\newcommand{\practica}[2]{%
    \title{Pr\'actica #1 \\ #2}
    \author{Algoritmos y Estructuras de Datos I}
    \date{Primer Cuatrimestre 2019}

    \maketitlepractica{#1}{#2}
}

\newcommand \maketitlepractica[2] {%
\begin{center}
\begin{tabular}{r cr}
 \begin{tabular}{c}
{\large\bf\textsf{\ Algoritmos y Estructuras de Datos I\ }}\\
Primer Cuatrimestre 2019\\
\title{\normalsize Gu\'ia Pr\'actica #1 \\ \textbf{#2}}\\
\@title
\end{tabular} &
\begin{tabular}{@{} p{1.6cm} @{}}
\includegraphics[width=1.6cm]{logodpt.jpg}
\end{tabular} &
\begin{tabular}{l @{}}
 \emph{Departamento de Computaci\'on} \\
 \emph{Facultad de Ciencias Exactas y Naturales} \\
 \emph{Universidad de Buenos Aires} \\
\end{tabular}
\end{tabular}
\end{center}

\bigskip
}


% Simbolos varios

\newcommand{\nat}{\ensuremath{\mathds{N}}}
\newcommand{\ent}{\ensuremath{\mathds{Z}}}
\newcommand{\float}{\ensuremath{\mathds{R}}}
\newcommand{\bool}{\ensuremath{\mathsf{Bool}}}
\newcommand{\True}{\ensuremath{\mathrm{true}}}
\newcommand{\False}{\ensuremath{\mathrm{false}}}
\newcommand{\Then}{\ensuremath{\rightarrow}}
\newcommand{\Iff}{\ensuremath{\leftrightarrow}}
\newcommand{\implica}{\ensuremath{\longrightarrow}}
\newcommand{\IfThenElse}[3]{\ensuremath{\mathsf{if}\ #1\ \mathsf{then}\ #2\ \mathsf{else}\ #3\ \mathsf{fi}}}
\newcommand{\In}{\textsf{in }}
\newcommand{\Out}{\textsf{out }}
\newcommand{\Inout}{\textsf{inout }}
\newcommand{\yLuego}{\land _L}
\newcommand{\oLuego}{\lor _L}
\newcommand{\implicaLuego}{\implica _L}
\newcommand{\existe}[3]{\ensuremath{(\exists #1:\ent) \ #2 \leq #1 < #3 \ }}
\newcommand{\paraTodo}[3]{\ensuremath{(\forall #1:\ent) \ #2 \leq #1 < #3 \ }}

% Símbolo para marcar los ejercicios importantes (estrellita)
\newcommand\importante{\raisebox{0.5pt}{\ensuremath{\bigstar}}}


\newcommand{\rango}[2]{[#1\twodots#2]}
\newcommand{\comp}[2]{[\,#1\,|\,#2\,]}

\newcommand{\rangoac}[2]{(#1\twodots#2]}
\newcommand{\rangoca}[2]{[#1\twodots#2)}
\newcommand{\rangoaa}[2]{(#1\twodots#2)}

%ejercicios
\newtheorem{exercise}{Ejercicio}
\newenvironment{ejercicio}[1][]{\begin{exercise}#1\rm}{\end{exercise} \vspace{0.2cm}}
\newenvironment{items}{\begin{enumerate}[a)]}{\end{enumerate}}
\newenvironment{subitems}{\begin{enumerate}[i)]}{\end{enumerate}}
\newcommand{\sugerencia}[1]{\noindent \textbf{Sugerencia:} #1}

\lstnewenvironment{code}{
    \lstset{% general command to set parameter(s)
        language=C++, basicstyle=\small\ttfamily, keywordstyle=\slshape,
        emph=[1]{tipo,usa}, emphstyle={[1]\sffamily\bfseries},
        morekeywords={tint,forn,forsn},
        basewidth={0.47em,0.40em},
        columns=fixed, fontadjust, resetmargins, xrightmargin=5pt, xleftmargin=15pt,
        flexiblecolumns=false, tabsize=2, breaklines, breakatwhitespace=false, extendedchars=true,
        numbers=left, numberstyle=\tiny, stepnumber=1, numbersep=9pt,
        frame=l, framesep=3pt,
    }
   \csname lst@SetFirstLabel\endcsname}
  {\csname lst@SaveFirstLabel\endcsname}


%tipos basicos
\newcommand{\rea}{\ensuremath{\mathsf{Float}}}
\newcommand{\cha}{\ensuremath{\mathsf{Char}}}
\newcommand{\str}{\ensuremath{\mathsf{String}}}

\newcommand{\mcd}{\mathrm{mcd}}
\newcommand{\prm}[1]{\ensuremath{\mathsf{prm}(#1)}}
\newcommand{\sgd}[1]{\ensuremath{\mathsf{sgd}(#1)}}

\newcommand{\tuple}[2]{\ensuremath{#1 \times #2}}

%listas
\newcommand{\TLista}[1]{\ensuremath{seq \langle #1\rangle}}
\newcommand{\lvacia}{\ensuremath{[\ ]}}
\newcommand{\lv}{\ensuremath{[\ ]}}
\newcommand{\longitud}[1]{\ensuremath{|#1|}}
\newcommand{\cons}[1]{\ensuremath{\mathsf{addFirst}}(#1)}
\newcommand{\indice}[1]{\ensuremath{\mathsf{indice}}(#1)}
\newcommand{\conc}[1]{\ensuremath{\mathsf{concat}}(#1)}
\newcommand{\cab}[1]{\ensuremath{\mathsf{head}}(#1)}
\newcommand{\cola}[1]{\ensuremath{\mathsf{tail}}(#1)}
\newcommand{\sub}[1]{\ensuremath{\mathsf{subseq}}(#1)}
\newcommand{\en}[1]{\ensuremath{\mathsf{en}}(#1)}
\newcommand{\cuenta}[2]{\mathsf{cuenta}\ensuremath{(#1, #2)}}
\newcommand{\suma}[1]{\mathsf{suma}(#1)}
\newcommand{\twodots}{\ensuremath{\mathrm{..}}}
\newcommand{\masmas}{\ensuremath{++}}
\newcommand{\matriz}[1]{\TLista{\TLista{#1}}}

% Acumulador
\newcommand{\acum}[1]{\ensuremath{\mathsf{acum}}(#1)}
\newcommand{\acumselec}[3]{\ensuremath{\mathrm{acum}(#1 |  #2, #3)}}

% \selector{variable}{dominio}
\newcommand{\selector}[2]{#1~\ensuremath{\leftarrow}~#2}
\newcommand{\selec}{\ensuremath{\leftarrow}}

\newcommand{\pred}[3]{%
    {\normalfont\bfseries\ttfamily pred }%
    {\normalfont\ttfamily #1}%
    \ifthenelse{\equal{#2}{}}{}{\ (#2) }%
    \{\ensuremath{#3}\}%
    {\normalfont\bfseries\,\par}%
  }

\newenvironment{proc}[4][res]{%
  % El parametro 1 (opcional) es el nombre del resultado
  % El parametro 2 es el nombre del problema
  % El parametro 3 son los parametros
  % El parametro 4 es el tipo del resultado
  % Preambulo del ambiente problema
  % Tenemos que definir los comandos requiere, asegura, modifica y aux
  \newcommand{\pre}[2][]{%
    {\normalfont\bfseries\ttfamily Pre}%
    \ifthenelse{\equal{##1}{}}{}{\ {\normalfont\ttfamily ##1} :}\ %
    \{\ensuremath{##2}\}%
    {\normalfont\bfseries\,\par}%
  }
  \newcommand{\post}[2][]{%
    {\normalfont\bfseries\ttfamily Post}%
    \ifthenelse{\equal{##1}{}}{}{\ {\normalfont\ttfamily ##1} :}\
    \{\ensuremath{##2}\}%
    {\normalfont\bfseries\,\par}%
  }
  \renewcommand{\aux}[4]{%
    {\normalfont\bfseries\ttfamily aux\ }%
    {\normalfont\ttfamily ##1}%
    \ifthenelse{\equal{##2}{}}{}{\ (##2)}\ : ##3\, = \ensuremath{##4}%
    {\normalfont\bfseries\,;\par}%
  }
  \newcommand{\res}{#1}
  \vspace{1ex}
  \noindent
  \encabezadoDeProc{#1}{#2}{#3}{#4}
  % Abrimos la llave
  \{\par%
  \tocarEspacios
}
% Ahora viene el cierre del ambiente problema
{
  % Cerramos la llave
  \noindent\}
  \vspace{1ex}
}


  \newcommand{\aux}[4]{%
    {\normalfont\bfseries\ttfamily aux\ }%
    {\normalfont\ttfamily #1}%
    \ifthenelse{\equal{#2}{}}{}{\ (#2)}\ : #3\, = \ensuremath{#4}%
    {\normalfont\bfseries\,;\par}%
  }


% \newcommand{\pre}[1]{\textsf{pre}\ensuremath{(#1)}}

\newcommand{\procnom}[1]{\textsf{#1}}
\newcommand{\procil}[3]{\textsf{proc #1}\ensuremath{(#2) = #3}}
\newcommand{\procilsinres}[2]{\textsf{proc #1}\ensuremath{(#2)}}
\newcommand{\preil}[2]{\textsf{Pre #1: }\ensuremath{#2}}
\newcommand{\postil}[2]{\textsf{Post #1: }\ensuremath{#2}}
\newcommand{\auxil}[2]{\textsf{fun }\ensuremath{#1 = #2}}
\newcommand{\auxilc}[4]{\textsf{fun }\ensuremath{#1( #2 ): #3 = #4}}
\newcommand{\auxnom}[1]{\textsf{fun }\ensuremath{#1}}
\newcommand{\auxpred}[3]{\textsf{pred }\ensuremath{#1( #2 ) \{ #3 \}}}

\newcommand{\comentario}[1]{{/*\ #1\ */}}

\newcommand{\nom}[1]{\ensuremath{\mathsf{#1}}}


% En las practicas/parciales usamos numeros arabigos para los ejercicios.
% Aca cambiamos los enumerate comunes para que usen letras y numeros
% romanos
\newcommand{\arreglarincisos}{%
  \renewcommand{\theenumi}{\alph{enumi}}
  \renewcommand{\theenumii}{\roman{enumii}}
  \renewcommand{\labelenumi}{\theenumi)}
  \renewcommand{\labelenumii}{\theenumii)}
}



%%%%%%%%%%%%%%%%%%%%%%%%%%%%%% PARCIAL %%%%%%%%%%%%%%%%%%%%%%%%
\let\@xa\expandafter
\newcommand{\tituloparcial}{\centerline{\depto -- \lamateria}
  \centerline{\elnombre -- \lafecha}%
  \setlength{\TPHorizModule}{10mm} % Fija las unidades de textpos
  \setlength{\TPVertModule}{\TPHorizModule} % Fija las unidades de
                                % textpos
  \arreglarincisos
  \newcounter{total}% Este contador va a guardar cuantos incisos hay
                    % en el parcial. Si un ejercicio no tiene incisos,
                    % cuenta como un inciso.
  \newcounter{contgrilla} % Para hacer ciclos
  \newcounter{columnainicial} % Se van a usar para los cline cuando un
  \newcounter{columnafinal}   % ejercicio tenga incisos.
  \newcommand{\primerafila}{}
  \newcommand{\segundafila}{}
  \newcommand{\rayitas}{} % Esto va a guardar los \cline de los
                          % ejercicios con incisos, asi queda mas bonito
  \newcommand{\anchode
  }{20} % Es para textpos
  \newcommand{\izquierda}{7} % Estos dos le dicen a textpos donde colocar
  \newcommand{\abajo}{2}     % la grilla
  \newcommand{\anchodecasilla}{0.4cm}
  \setcounter{columnainicial}{1}
  \setcounter{total}{0}
  \newcounter{ejercicio}
  \setcounter{ejercicio}{0}
  \renewenvironment{ejercicio}[1]
  {%
    \stepcounter{ejercicio}\textbf{\noindent Ejercicio \theejercicio. [##1
      puntos]}% Formato
    \renewcommand\@currentlabel{\theejercicio}% Esto es para las
                                % referencias
    \newcommand{\invariante}[2]{%
      {\normalfont\bfseries\ttfamily invariante}%
      \ ####1\hspace{1em}####2%
    }%
    \newcommand{\Proc}[5][result]{
      \encabezadoDeProc{####1}{####2}{####3}{####4}\hspace{1em}####5}%
  }% Aca se termina el principio del ejercicio
  {% Ahora viene el final
    % Esto suma la cantidad de incisos o 1 si no hubo ninguno
    \ifthenelse{\equal{\value{enumi}}{0}}
    {\addtocounter{total}{1}}
    {\addtocounter{total}{\value{enumi}}}
    \ifthenelse{\equal{\value{ejercicio}}{1}}{}
    {
      \g@addto@macro\primerafila{&} % Si no estoy en el primer ej.
      \g@addto@macro\segundafila{&}
    }
    \ifthenelse{\equal{\value{enumi}}{0}}
    {% No tiene incisos
      \g@addto@macro\primerafila{\multicolumn{1}{|c|}}
      \bgroup% avoid overwriting somebody else's value of \tmp@a
      \protected@edef\tmp@a{\theejercicio}% expand as far as we can
      \@xa\g@addto@macro\@xa\primerafila\@xa{\tmp@a}%
      \egroup% restore old value of \tmp@a, effect of \g@addto.. is

      \stepcounter{columnainicial}
    }
    {% Tiene incisos
      % Primero ponemos el encabezado
      \g@addto@macro\primerafila{\multicolumn}% Ahora el numero de items
      \bgroup% avoid overwriting somebody else's value of \tmp@a
      \protected@edef\tmp@a{\arabic{enumi}}% expand as far as we can
      \@xa\g@addto@macro\@xa\primerafila\@xa{\tmp@a}%
      \egroup% restore old value of \tmp@a, effect of \g@addto.. is
      % global
      % Ahora el formato
      \g@addto@macro\primerafila{{|c|}}%
      % Ahora el numero de ejercicio
      \bgroup% avoid overwriting somebody else's value of \tmp@a
      \protected@edef\tmp@a{\theejercicio}% expand as far as we can
      \@xa\g@addto@macro\@xa\primerafila\@xa{\tmp@a}%
      \egroup% restore old value of \tmp@a, effect of \g@addto.. is
      % global
      % Ahora armamos la segunda fila
      \g@addto@macro\segundafila{\multicolumn{1}{|c|}{a}}%
      \setcounter{contgrilla}{1}
      \whiledo{\value{contgrilla}<\value{enumi}}
      {%
        \stepcounter{contgrilla}
        \g@addto@macro\segundafila{&\multicolumn{1}{|c|}}
        \bgroup% avoid overwriting somebody else's value of \tmp@a
        \protected@edef\tmp@a{\alph{contgrilla}}% expand as far as we can
        \@xa\g@addto@macro\@xa\segundafila\@xa{\tmp@a}%
        \egroup% restore old value of \tmp@a, effect of \g@addto.. is
        % global
      }
      % Ahora armo las rayitas
      \setcounter{columnafinal}{\value{columnainicial}}
      \addtocounter{columnafinal}{-1}
      \addtocounter{columnafinal}{\value{enumi}}
      \bgroup% avoid overwriting somebody else's value of \tmp@a
      \protected@edef\tmp@a{\noexpand\cline{%
          \thecolumnainicial-\thecolumnafinal}}%
      \@xa\g@addto@macro\@xa\rayitas\@xa{\tmp@a}%
      \egroup% restore old value of \tmp@a, effect of \g@addto.. is
      \setcounter{columnainicial}{\value{columnafinal}}
      \stepcounter{columnainicial}
    }
    \setcounter{enumi}{0}%
    \vspace{0.2cm}%
  }%
  \newcommand{\tercerafila}{}
  \newcommand{\armartercerafila}{
    \setcounter{contgrilla}{1}
    \whiledo{\value{contgrilla}<\value{total}}
    {\stepcounter{contgrilla}\g@addto@macro\tercerafila{&}}
  }
  \newcommand{\grilla}{%
    \g@addto@macro\primerafila{&\textbf{TOTAL}}
    \g@addto@macro\segundafila{&}
    \g@addto@macro\tercerafila{&}
    \armartercerafila
    \ifthenelse{\equal{\value{total}}{\value{ejercicio}}}
    {% No hubo incisos
      \begin{textblock}{\anchodegrilla}(\izquierda,\abajo)
        \begin{tabular}{|*{\value{total}}{p{\anchodecasilla}|}c|}
          \hline
          \primerafila\\
          \hline
          \tercerafila\\
          \tercerafila\\
          \hline
        \end{tabular}
      \end{textblock}
    }
    {% Hubo incisos
      \begin{textblock}{\anchodegrilla}(\izquierda,\abajo)
        \begin{tabular}{|*{\value{total}}{p{\anchodecasilla}|}c|}
          \hline
          \primerafila\\
          \rayitas
          \segundafila\\
          \hline
          \tercerafila\\
          \tercerafila\\
          \hline
        \end{tabular}
      \end{textblock}
    }
  }%
  \vspace{0.4cm}
  \textbf{Nro. de orden:}

  \textbf{LU:}

  \textbf{Apellidos:}

  \textbf{Nombres:}
  
  \textbf{Nro. de hojas que adjunta:}
  \vspace{0.5cm}
}



% AMBIENTE CONSIGNAS
% Se usa en el TP para ir agregando las cosas que tienen que resolver
% los alumnos.
% Dentro del ambiente hay que usar \item para cada consigna

\newcounter{consigna}
\setcounter{consigna}{0}

\newenvironment{consignas}{%
  \newcommand{\consigna}{\stepcounter{consigna}\textbf{\theconsigna.}}%
  \renewcommand{\ejercicio}[1]{\item ##1 }
  \renewcommand{\proc}[5][result]{\item
    \encabezadoDeProc{##1}{##2}{##3}{##4}\hspace{1em}##5}%
  \newcommand{\invariante}[2]{\item%
    {\normalfont\bfseries\ttfamily invariante}%
    \ ##1\hspace{1em}##2%
  }
  \renewcommand{\aux}[4]{\item%
    {\normalfont\bfseries\ttfamily aux\ }%
    {\normalfont\ttfamily ##1}%
    \ifthenelse{\equal{##2}{}}{}{\ (##2)}\ : ##3 \hspace{1em}##4%
  }
  % Comienza la lista de consignas
  \begin{list}{\consigna}{%
      \setlength{\itemsep}{0.5em}%
      \setlength{\parsep}{0cm}%
    }
}%
{\end{list}}



% para decidir si usar && o ^
\newcommand{\y}[0]{\ensuremath{\land}}

% macros de correctitud
\newcommand{\semanticComment}[2]{#1 \ensuremath{#2};}
\newcommand{\namedSemanticComment}[3]{#1 #2: \ensuremath{#3};}


\newcommand{\local}[1]{\semanticComment{local}{#1}}

\newcommand{\vale}[1]{\semanticComment{vale}{#1}}
\newcommand{\valeN}[2]{\namedSemanticComment{vale}{#1}{#2}}
\newcommand{\impl}[1]{\semanticComment{implica}{#1}}
\newcommand{\implN}[2]{\namedSemanticComment{implica}{#1}{#2}}
\newcommand{\estado}[1]{\semanticComment{estado}{#1}}

\newcommand{\invarianteCN}[2]{\namedSemanticComment{invariante}{#1}{#2}}
\newcommand{\invarianteC}[1]{\semanticComment{invariante}{#1}}
\newcommand{\varianteCN}[2]{\namedSemanticComment{variante}{#1}{#2}}
\newcommand{\varianteC}[1]{\semanticComment{variante}{#1}}

\usepackage{caratula} % Version modificada para usar las macros de algo1 de ~> https://github.com/bcardiff/dc-tex
\usepackage{amsfonts} 
\usepackage[utf8]{inputenc}

\begin{document}

\titulo{TP de Especificaci\'on}
\subtitulo{An\'alisis Habitacional Argentino}
\fecha{8 de Septiembre de 2021}
\materia{Algoritmos y Estructuras de Datos I}
\grupo{Grupo 7}
\newcommand{\senial}{\textit{se\~nal}}

% Pongan cuantos integrantes quieran
\integrante{Acuña, Martín}{596/21}{acunamartin1426@gmail.com}
\integrante{Castro, Lucía}{278/21}{lucia.ines.castro.98@gmail.com}
\integrante{Clas, Giulia}{11/15}{clas.giulia.s@gmail.com}
\integrante{Seidler, Daniel}{973/12}{danieljseidler@gmail.com}

\maketitle

\section{Problemas}

% ej 1
\begin{proc}{encuestaV\'alida}{\In th: $eph_h$, \In ti: $eph_i$, \Out result: \bool}{}
    \pre{\True}
    \post{result =\True\Iff laEncuestaEsValida(th,ti)}
\end{proc}

% ej 2
\begin{proc}{histHabitacional}{\In th: $eph_{h}$, \In ti: $eph_{i}$, \In region: \ent, \Out res: \TLista{\ent} }{}
    \pre{laEncuentaEsValida(th, ti) \land laRegionEsValida(region)}
    \post{(\forall i :\ent)(0 \leq i < \longitud{res} \implicaLuego 
    \\res[i] = casasConNHabitaciones(th, region, i+1)) \land esMaxDeHabitaciones(\longitud{res}, th, region)}
\end{proc}

% ej 3
\begin{proc}{laCasaEstaQuedandoChica}{\In th: $eph_h$, \In ti: $eph_i$, \Out res: $seq\langle\mathbb{R}\rangle$}{}
    \pre{laEncuentaEsValida(th, ti)}
    \post{(\forall i :\ent)(0 \leq i < \longitud{res} \implicaLuego \\
    res[i] = \frac{casasConHacinamientoCriticoEnLaRegion(th, ti, i)}{hogaresEnLaRegion(th,i)}}
\end{proc}

% ej 4
\begin{proc}{creceElTeleworkingEnCiudadesGrandes}{\In t1h: $eph_h$, \In t1i: $eph_i$, \In t2h: $eph_h$, \In t2i: $eph_i$, \Out result: \bool}{}
    \pre{(laEncuestaEsValida(t1h, t1i) \text{ } \land  \text{ } laEncuestaEsValida(t2h, t2i)) \text{ } \land_\text{ } L (t1h[0][ord(HOGANO)] < \\
t2h[0][ord(HOGANO)] \land_L t1h[0][ord(HOGTRIMESTRE)] = t2h[0][ord(HOGTRIMESTRE)] }
    \post{result = \\ 
\frac{cantidadPersonasQueHacenTeleworkingEnCiudadGrande(t1h, t1i)}{cantidadPersonasEnCiudadGrande(t1h, t1i)} <  \frac{cantidadPersonasQueHacenTeleworkingEnCiudadGrande(t2h, t2i)}{cantidadPersonasEnCiudadGrande(t2h, t2i)}}
\end{proc}

% ej 5
\begin{proc}{costoSubsidioMejora}{\In th: $eph_{h}$, \In ti: $eph_{i}$, \In monto: \ent, \Out res: \ent }{}
    \pre{laEncuentaEsValida(th, ti) \land monto > 0}
    \post{res = monto * hogaresCandidatosASubsidio(th,ti)}
\end{proc}

% ej 6
\begin{proc}{generarJoin}{\In th: $eph_{h}$,\In ti: $eph_{i}$, \Inout junta: $joinHI$}{}
    \pre{laEncuestaEsValida(th,ti)}
    \post{(\forall i:\ent)(0\leq i<filas(junta)\implicaLuego es2-upla(junta[i]))}
\end{proc}

% ej  7
\begin{proc}{ordenarRegionYTipo}{\Inout th: $eph_{h}$, \Inout ti: $eph_{i}$}{}
    \pre{laEncuentaEsValida(th, ti) \land th = th_{0} \land ti = ti_{0}}
    \post{mismoTamanioDeTabla(th, th_{0}) \land mismoTamanioDeTabla(ti, ti_{0}) \land \\ 
    	th \subseteq th_{0} \land ti \subseteq ti_{0} \land ordenadoPorRegion(th) \land \\ regionesOrdenadasPorHogcodusu(th) \land mismoOrdenDeCodusu(th, ti) \land casasOrdenadasPorComponente(ti)}
\end{proc}

% ej 8
\begin{proc}{muestraHomogenea}
{\In th: $eph_{h}$, \In ti: $eph_{i}$ \Out res: \TLista{hogar} }{}
    \pre{laEncuentaEsValida(th, ti)}
    \post{(\,estaOrdenadaPorIngresos(ti, res) \\ \land                         
    	   diferenciaDeIngresosConstante(ti,res) \\ \land
           \longitud{res} \geq 3 \\ \land
            esLaSecuenciaMasGrande(th, ti, res) ) \\ \lor
            \longitud{res} = 0
    }
\end{proc}

% ej 9
\begin{proc}{corregirRegion}{\Inout th: $eph_h$, \In ti: $eph_i$}{}
    \pre{laEncuentaEsValida(th, ti) \land th_0=th}
    \post{\longitud{th_0}=\longitud{th} \land cantColumnasHogares(th_0)=cantColumnasHogares(th) \land (\forall i :\ent)(\forall col :\ent)\\
        (0 \leq i < \longitud{th} \land col \neq ord(REGION) \implicaLuego th_0[i][col]=th[i][col]) \land\\
	(0 \leq i < \longitud{th} \land th_0[i][ord(REGION)] = 1 \implicaLuego th[i][ord(REGION)]=5) \land \\
	(0 \leq i < \longitud{th} \land th_0[i][ord(REGION)] \neq 1 \implicaLuego th_0[i][ord(REGION)]=th[i][ord(REGION)])}
\end{proc}

% ej 10
\begin{proc}{histoGramaDeAnillosConcentricos}
{\In th: $eph_{h}$, \In centro: \ent x\ent, \In distancias: \TLista{\ent} \Out result: \TLista{\ent} }{}
    \pre{esListaOrdenadaCreciente(distancias) \\ \land
        \longitud{distancias}>0 \\ \land
         noTieneValoresNulosNiNegativos(distancias)}
    \post{ |res|=|distancias| \\
    	\land res[0]=cantidadDeHogaresEntreDosDistancias(th, centro, 0, distancias[0]) \,\\
	\land (\forall i:\ent)(1 \leq i < \longitud{res} \implicaLuego \\
        res[i]=cantidadDeHogaresEntreDosDistancias(th, centro, distancias[i-1], distancias[i]) \\
    }
\end{proc}

% ej 11
\begin{proc}{quitarIndividuos}{\Inout th: $eph_h$, \Inout ti: $eph_i$, \In busqueda: seq$\langle(ItemIndividuo, dato)\rangle$, \Out result: ($eph_h,eph_i$)}{}
    \pre{laEncuentaEsValida(th, ti) \land th_0=th \land ti_0=ti \land busquedaValida(busqueda)}
    \post{result_0 \subseteq th_0 \land result_1 \subseteq ti_0 \land interseccionVacia(result_0, th) \land interseccionVacia(result, ti) \land \\
    (\forall i: \ent) (0 \leq i < |result_1| \implicaLuego cumpleBusqueda(result[i], busqueda) \land \\
    (\forall i: \ent) (0 \leq i < |ti| \implicaLuego \neg cumpleBusqueda(ti[i], busqueda) \land \\
    (\forall i: \ent) (0 \leq i < |result_1| \implicaLuego contieneHogarDeIndividuo(result[i], result_0) \land \\
    \neg contieneHogarDeIndividuo(result[i], th)\\
   (\forall i: \ent) (0 \leq i < |result_1| \implicaLuego contieneHogarDeIndividuo(ti[i], th) \land \neg contieneHogarDeIndividuo(ti[i], result_0) \\
}
\end{proc}


\section{Predicados y Auxiliares generales}
Al comienzo hay algunos predicados genéricos. En adelante, se separaron los predicados y auxiliares en orden en que fueron apareciendo, pero se reutilizan algunos en ejercicios posteriores.

% Generales
\subsection{Predicados generales}
\subparagraph{}
\pred{pertenece}{elem: T, s: seq $\langle T \rangle$}{\\\indent\indent\indent
	(\exists i : \ent)(0 \leq i < |s| \land_L elem=s[i]) \\\indent\indent
}

\subparagraph{}
\pred{estaContenido}{$t: seq\langle seq \langle dato \rangle \rangle$, $t_{0}: seq \langle seq \langle dato \rangle \rangle$}{\\ \indent\indent\indent
	(\forall i :\ent)(0 \leq i < filas(t) \implicaLuego ((\exists j :\ent)(0 \leq j < filas(t_{0}) \land_L (t[i]=t_{0}[j]))) \\ \indent\indent
}


% ej 1
\subsection{Ejercicio 1}
\subparagraph{}
\pred{laEncuestaEsValida}{th: $eph_{h}$, ti: $eph_{i}$}{\\ \indent\indent\indent
	esMatriz(th)\,\wedge\,esMatriz(ti)\,\wedge\,esTablaNoVacia(th)\,\wedge\,esTablaNoVacia(ti)\,\wedge\\\indent\indent\indent cantColumnasHogaress(th)\,\wedge\,
	cantColumnasIndividuos(ti)\,\wedge\\\indent\indent\indent aCadaHogarLeCorrespondeUnIndividuo(th, ti)\,\wedge\,aCadaIndividuoLeCorrespondeUnHogar(th, ti)\,\wedge\\\indent\indent\indent noHayRepetidos(th, ord(HOGCODUSU))\,\wedge\,noHayRepetidos(ti, ord(INDCODUSU))\,\wedge\\\indent\indent\indent latitudLongitudValidas(th)\,\wedge\,anioTrimestreTabla(th,ord(HOGANO),ord(HOGTRIMESTRE))\,\wedge\\\indent\indent\indent anioTrimestreTabla(tI,ord(INDANO),ord(INDTRIMESTRE))\,\wedge\,cantMiembrosHogarMenorOIgual20(th, ti)\,\wedge\,
	\indent\indent\indent atributoIV2MayoroIgualII2(th)\,\wedge\,atributosHogarRangoEsperado(th)\,\wedge\, atributosIndividuoRangoEsperado(ti) 
\\\indent\indent }

\subparagraph{}
\pred{esMatriz}{m: \matriz {dato}}{\\\indent\indent\indent (\forall i:\ent)(0\leq i<filas(m)\implicaLuego |m(i)|>0\,\wedge\,(\forall j:\ent)(0\leq j<filas(m)\implicaLuego |m(i)|=|m(j)|)) \\\indent\indent }

\subparagraph{}
\aux{filas}{m: \matriz {dato}}{\ent}{|m| }

\subparagraph{}
\aux{columnas}{m: \matriz {dato}}{\ent}{\IfThenElse {filas(m) > 0}{|m(0)|}{0} }

\subparagraph{}
\pred{esTablaNoVacia}{m: \matriz {dato}}{\\\indent\indent\indent |m|\neq 0 \\\indent\indent }

\subparagraph{}
\pred{cantColumnasHogares}{m: $eph_{h}$}{\\\indent\indent\indent (\forall i:\ent)(0\leq i < columnas(m)\implicaLuego ItemHogar(i)\neq \perp) \wedge (\forall i:\ent)( i\geq columnas(m)\implicaLuego \\\indent\indent\indent ItemHogar(i)= \perp)
\\\indent\indent }

\subparagraph{}
\pred{cantColumnasIndividuos}{m: $eph_{i}$}{\\\indent\indent\indent (\forall i:\ent)(0\leq i < columnas(m) \implicaLuego ItemIndividuo(i)\neq \perp) \wedge (\forall i:\ent)(i\geq columnas(m)\implicaLuego \\\indent\indent\indent ItemIndividuo(i)= \perp)
\\\indent\indent }

\subparagraph{}
\pred{aCadaHogarLeCorrespondeUnIndividuo}{m: $eph_{h}$, z: $eph_{i}$}{\\\indent\indent\indent (\forall i:\ent)(0\leq i<filas(m)\implicaLuego (\exists j:\ent)(0\leq j<filas(z)\,\yLuego\, z[j][ord(INDCODUSU)] = m[i][ord(HOGCODUSU)])\\\indent\indent}

\subparagraph{}
\pred{aCadaIndividuoLeCorrespondeUnHogar}{m: $eph_{h}$, z: $eph_{i}$}{\\\indent\indent\indent (\forall j:\ent)(0\leq j<filas(z)\implicaLuego (\exists i:\ent)(0\leq i<filas(m)\,\yLuego\, z[j][ord(INDCODUSU)] = m[i][ord(HOGCODUSU)])\\\indent\indent}

\subparagraph{}
\pred{noHayRepetidos}{m: \matriz {dato}, l:\ent}{\\\indent\indent\indent (\forall i:\ent)(0\leq i<filas(m)\implicaLuego (\forall j:\ent)((0\leq j<filas(m)\,\wedge\,j\neq i)\implicaLuego m[i][l]\neq m[j][l])
\\\indent\indent }

\subparagraph{}
\pred{latitudLongitudValidas}{m: $eph_{h}$}{\\\indent\indent\indent (\forall i:\ent)(0\leq i<filas(m)\implicaLuego (-55\leq m[i][ord(HOGLATITUD)]\leq-22))\,\wedge\\\indent\indent\indent(\forall j:\ent)(0\leq i<filas(m)\implicaLuego(-74\leq m[J][ord(HOGLONGITUD)]\leq-53))  
\\\indent\indent }
%Fuente consultada para las coordenadas: https://www.ecured.cu/Anexo:Regiones_geogr%C3%A1ficas_de_Argentina#:~:text=Las%20regiones%20son%20ocho%3A%20la,Patagonia%20y%20las%20Tierras%20Australes.

\subparagraph{}
\aux{anoEncuesta}{m: $eph_{h}$}{\ent}{m[0][ord(HOGANO] }

\subparagraph{}
\aux{trimestreEncuesta}{m: $eph_{h}$}{\ent}{m[0][ord(HOGTRIMESTRE)] }

\subparagraph{}
\pred{anioTrimestreTabla}{m: \matriz {dato}, l,r:\,\ent}{\\\indent\indent\indent (\forall i:\ent)(0\leq i<filas(m)\implicaLuego m[i][l] = anoEncuesta\,\wedge\, m[i][r] = trimestreEncuesta) 
\\\indent\indent }

\subparagraph{}
\pred{cantMiembrosHogarMenorOIgual20}{m: $eph_{h}$, z: $eph_{i}$}{\\\indent\indent\indent (\forall i:\ent)(0\leq i<filas(m)\implicaLuego (\sum_{j=0}^{filas(z)-1}{\IfThenElse {hayCorrespondenciaHogarIndividuo (i,j,m,z)}{1}{0} \leq 20)) } \\\indent\indent }

\subparagraph{}
\pred{atributoIV2MayoroIgualII2}{m: $eph_{h}$}{\\\indent\indent\indent (\forall i:\ent)(0\leq i<filas(m)\implicaLuego m[i][ord(IV2)]\geq m[I][ord(II2)])\\\indent\indent }

\subparagraph{}
\pred{atributosHogarRangoEsperado}{m: $eph_{h}$}{\\\indent\indent\indent (\forall i:\ent)(0\leq i<filas(m)\implicaLuego ((0<m[i][ord(HOGCODUSU)])\,\wedge\,(0<m[i][ord(HOGANO)])\,\wedge\\\indent\indent\indent(0<m[i][ord(HOGTRIMESTRE)])\,\wedge\,(1\leq m[i][ord(II7)]\leq 3)\,\wedge\,(1\leq m[i][ord(REGION)]\leq 10)\,\wedge\\\indent\indent\indent(0\leq m[i][ord(MAS\_500)]\leq 1)\,\wedge\,(1\leq m[i][ord(IV1)]\leq 5)\,\wedge\,(1\leq m[i][ord(II3)]\leq 3)))\\\indent\indent }

\subparagraph{}
\pred{atributosIndividuoRangoEsperado}{z: $eph_{i}$}{\\\indent\indent\indent (\forall i:\ent)(0\leq i<filas(m)\implicaLuego ((0<z[i][ord(INDCODUSU)])\,\wedge\,(0<z[i][ord(INDANO)])\,\wedge\\\indent\indent\indent(0<z[i][ord(INDTRIMESTRE)])\,\wedge\,(0<z[i][ord(COMPONENTE)])\,\wedge\,(0\leq z[i][ord(CH6)])\,\wedge\\\indent\indent\indent(-1\leq z[i][ord(P47T)])\,\wedge\,(1\leq z[i][ord(CH4)]\leq 2)\,\wedge\,(0\leq z[i][ord(NIVEL_ED)]\leq 1)\,\wedge\\\indent\indent\indent(-1\leq Z[i][ord(ESTADO)]\leq 1)\,\wedge\,(0\leq z[i][ord(CAT_OCUP)]\leq 4)\,\wedge\,(1\leq z[i][ord(PP04G)]\leq 10)))\\\indent\indent }

% ej 2
\subsection{Ejercicio 2}
\subparagraph{}
\aux{casasConNHabitaciones}{th: $eph_{h}$, region: \ent, n: \ent}{\ent}{\\ \indent\indent\indent
	\sum_{i=0}^{\longitud{th}-1} if \ (th[i][ord(region)]=region) \land esCasa(th[i]) \land (th[i][ord(IV2)]=n) \ then \ 1 \ else \ 0 \ fi
}

\subparagraph{}
\pred{laRegionEsValida}{region: \ent}{\\ \indent\indent\indent
	0 < region \leq 6 \\ \indent\indent
}

\subparagraph{}
\pred{esMaxDeHabitaciones}{th: $eph_{h}$, region: \ent, n: \ent}{\\ \indent\indent\indent
	(\forall i :\ent)((0 \leq i < filas(th) \land th[i][ord(REGION)]=region) \implicaLuego th[i][ord(IV2)] \leq n) \\ \indent\indent
}   

% ej 3
\subsection{Ejercicio 3}
\subparagraph{}
\aux{casasConHacinamientoCriticoEnLaRegion}{th: $eph_{h}$, ti: $eph_{i}$, i: \ent}{\ent}{\\ \sum_{j=0}^{\longitud{th}-1} if \ (th[j][ord(REGION)]=i) \land (esCasaConHacinamientoCritico(th, ti, j) \ then \ 1 \ else \ 0 \ fi}

\subparagraph{}
\pred{esCasaConHacinamientoCritico}{th: $eph_{h}$, ti: $eph_{i}$, i: \ent}{ \\
\indent\indent\indent esCasa(th[i]) \land hogarEnCiudadGrande(th[i])  \land esHogarConHacinamientoCritico(th, ti, i) \\
\indent\indent}

\subparagraph{}
\pred{esCasa}{h: hogar}{ \\
\indent\indent\indent h[ord(IV1)]=1 \\ 
\indent\indent}

\subparagraph{}
\pred{hogarEnCiudadGrande}{h: hogar}{ \\ 
	\indent\indent\indent h[ord(MAS\_500)] = 1 \\ 
\indent\indent}


\subparagraph{}
\pred{esHogarConHacinamientoCritico}{th: $eph_{h}$, ti: $eph_{i}$, i: \ent}{ \\
\indent\indent\indent \frac{miembrosDelHogar(ti,th[i][ord(CODUSU)]}{cantidadDeHabitaciones(th[i])}>3 \\ 
\indent\indent}

\subparagraph{}
\aux{miembrosDelHogar}{ti: $eph_{i}$, codusu: dato }{\ent}{\\ \sum_{j=0}^{\longitud{ti}-1} if \ ti[j][ord(CODUSU)]=codusu \ then \ 1 \ else \ 0 \ fi}

\subparagraph{}
\aux{cantidadDeHabitaciones}{h: hogar}{\ent}{\\ h[ord(IV2)]}

\subparagraph{}
\aux{hogaresEnLaRegion}{th: $eph_{h}$, i: \ent}{\ent}{\\ \sum_{j=0}^{\longitud{th}-1} if \ th[j][ord(REGION)]=i \ then \ 1 \ else \ 0 \ fi}

% ej 4
\subsection{Ejercicio 4}

\subparagraph{}
\pred{esCasaODepartamento}{h: hogar}{\\ 
\indent\indent\indent h[ord(IV1)] = 1 \lor h[ord(IV1)] = 2 \\
\indent\indent}

\subparagraph{}
\pred{haceTeleworking}{h: hogar, i: individuo}{\\ 
\indent\indent\indent h[ord(II3)] = 1 \land i[ord(PP04G)] = 6\\
\indent\indent}

\subparagraph{}
\aux{cantidadPersonasEnHogarEnCiudadGrande}{h: hogar, ti: $eph_i$}{\ent}{\\ 
\indent\indent\indent \sum_{i=0}^{|ti|-1} if \text{ hogarEnCiudadGrande(h) } \land \text{ h[ord(HOGCODUSU)] = ti[i][ord(INDCODUSU)] then 1 else 0 fi}
\indent\indent}

\subparagraph{}
\aux{cantidadPersonasQueHacenTeleworkingEnHogarEnCiudadGrande}{h: hogar, ti: $eph_i$}{\ent}{\\ \indent\indent\indent 
	\sum_{i=0}^{|ti|-1} \text{if
	 hogarEnCiudadGrande(h) } \land \text{ h[ord(HOGCODUSU)] = ti[i][ord(INDCODUSU)]} \land_L \text{ haceTeleworking(h, ti[i]) } \land_L \text{ esCasaODepartamento(h) then 1 else 0 fi}
}

\subparagraph{}
\aux{cantidadPersonasEnCiudadGrande}{th: $eph_h$, ti: $eph_i$}{\ent}{\\ \indent\indent\indent 
	\sum_{i=0}^{|th|-1}
	\text{cantidadPersonasEnHogarEnCiudadGrande(th[i], ti) }
}

\subparagraph{}
\aux{cantidadPersonasQueHacenTeleworkingEnCiudadGrande}{th: $eph_h$, ti: $eph_i$}{\ent}{\\ \indent\indent\indent 
	\sum_{i=0}^{|th|-1 } \text{cantidadPersonasQueHacenTeleworkingEnHogarEnCiudadGrande(th[i], ti) }
}

% ej 5
\subsection{Ejercicio 5}
\subparagraph{}
\pred{tenenciaPropia}{h: hogar}{ \\
\indent\indent\indent h[ord(II7)]=1 \\ 
\indent\indent}

\subparagraph{}
\aux{habitacionesParaDormir}{h: hogar}{\ent}{\\ \indent\indent\indent 
	h[ord(II2)]
}

\subparagraph{}
\aux{hogaresCandidatosASubsidio}{\In th: $eph_{h}$, \In ti: $eph_{i}$}{\ent}{\\ \indent\indent\indent
	\sum_{i=0}^{\longitud{th}-1} if \ esCasa(th[i]) \wedge tenenciaPropia(th[i]) \land (habitacionesParaDormir(th[i]) < \\ (miembrosDelHogar(ti,th[i][ord(HOGCODUSU)]) - 2)) \ then \ 1 \ else \ 0 \ fi}



% ej 6
\subsection{Ejercicio 6}
\subparagraph{}
\pred{es2-upla}{tupla: $hogar\,\times \,individuo$}{\\ \indent\indent\indent
	(tupla)_{0}[ord(HOGCODUSU)] = (tupla)_{1}[ord(INDCODUSU)]\,\wedge\, (\forall i:\ent)(i\geq 2\implicaLuego (tupla)_{i}= \perp) \\ \indent\indent
}

% ej 7
\subsection{Ejercicio 7}
\subparagraph{}
\pred{ordenadoPorRegion}{th: $eph_{h}$}{\\ \indent\indent\indent
	(\forall i :\ent)(0 \leq i < filas(th)-1 \implicaLuego (th[i][ord(REGION)] \leq th[i+1][ord(REGION)])) \\ \indent\indent
}

\subparagraph{}
\pred{regionesOrdenadasPorHogcodusu}{th: $eph_{h}$}{\\ \indent\indent\indent
	(\forall i :\ent)((0 \leq i < filas(th)-1 \,\land \, th[i][ord(REGION)] = th[i+1][ord(REGION)]) \implicaLuego (th[i][ord(HOGCODUSU)] \leq \\ \indent\indent\indent
	th[i+1][ord(HOGCODUSU)]) \\ \indent\indent
}

\subparagraph{}
\pred{mismoOrdenDeCodusu}{th: $eph_{h}$, ti: $eph_{i}$}{\\ \indent\indent\indent
	(\forall i :\ent)(0 \leq i < filas(th)-1 \implicaLuego (th[i][ord(HOGCODUSU)] = ti[i][ord(INDCODUSU)]) \\ \indent\indent
}

\subparagraph{}
\pred{casasOrdenadasPorComponente}{ti: $eph_{i}$}{\\ \indent\indent\indent
	(\forall i :\ent)((0 \leq i < filas(ti)-1 \land ti[i][ord(INDCODUSU)] = ti[i+1][ord(INDCODUSU)]) \implicaLuego \\  \indent\indent\indent
	(ti[i][ord(COMPONENTE)] \leq ti[i+1][ord(COMPONENTE)]) \\ \indent\indent
}

\subparagraph{}
\pred{mismoTamanioDeTabla}{$t: seq\langle seq\langle dato \rangle \rangle$, $t_{0}: seq \langle seq\langle dato \rangle \rangle$}{\\ \indent\indent\indent
filas(t) = filas(t_{0}) \land columnas(t) = columnas(t_{0}) \\ \indent\indent
}

% ej 8
\subsection{Ejercicio 8}
\subparagraph{}
\pred{estaOrdenadaPorIngresos}
	{\In th: $eph_{h}$, \In ti: $eph_{i}$ \In res: \TLista{hogar} }{ \\\indent\indent\indent 
	(\forall i:\ent)(0 \leq i < \longitud{res}-1 \implicaLuego
	\\\indent\indent\indent
	cantidadDeIngresos(ti, res[i][ord(HOGCODUSU)] \leq 
	cantidadDeIngresos(ti, res[i+1][ord(HOGCODUSU)]\\
\indent\indent}

\subparagraph{}
\aux{cantidadDeIngresos}
	{\In ti: $eph_{i}$, \In codusu : \ent}{\ent}{\\\indent\indent\indent
	\sum_{i=0}^{\longitud{ti}-1} if \, th[i][ord(INDCODUSU)] = codusu \wedge (t[i][ord(P47T)] \neq -1 \, then \, ti[i][ord(P47T)] \, else \, 0 \, fi 
}

\subparagraph{}
\pred{diferenciaDeIngresosConstante}
	{\In ti: $eph_{i}$ \In res: \TLista{hogar}}{\\\indent\indent\indent
	(\forall i:\ent)(0 \leq i < \longitud{res}-2 \implicaLuego 
	\\\indent\indent\indent
	cantidadDeIngresos(ti, res[i+1][ord(HOGCODUSU)] - cantidadDeIngresos(ti, res[i][ord(HOGCODUSU)]= 
	\\\indent\indent\indent
	 cantidadDeIngresos(ti, res[i+2][ord(HOGCODUSU)] - cantidadDeIngresos(ti, res[i+1][ord(HOGCODUSU)] \\
\indent\indent}

\subparagraph{}
\pred{esLaSecuenciaMasGrande}
	{\In th: $eph_{h}$, \In ti: $eph_{i}$, \In res: \TLista{hogar} }
	{\\\indent\indent\indent
	(\forall s: \TLista{hogar}) ((s \subseteq th) \wedge estaOrdenadaPorIngresos(ti, s) 
	\\\indent\indent\indent
	\wedge diferenciaDeIngresosConstante(ti, s)) \Then 
	\longitud{s} \leq \longitud{res}\\
}

% ej 9
\subsection{Ejercicio 9}
\subparagraph{}
No se utilizaron predicados auxiliares

% ej 10
\subsection{Ejercicio 10}
\subparagraph{}
\aux{cantidadDeHogaresEntreDosDistancias \\\indent\indent\indent}
{\In th: $eph_{h}$, \In centro : \ent x\ent, \In limiteInferior : \ent, \In limiteSuperior: \ent}
{\ent}
{\\\indent\indent\indent
	\sum_{i=0}^{\longitud{th}-1} if \,( limiteInferior \leq
	\\\indent\indent\indent\indent\indent\indent
	distanciaEuclidiana(th[i][ord(HOGLATITUD)], th[i][ord(HOGLONGITUD), centro_0, centro_1) <
	\\\indent\indent\indent\indent\indent\indent
	limiteSuperior) \, then \, 1 \, else \, 0 \, fi
}

\subparagraph{}
\aux{distanciaEuclidiana}{\In C1Lat: \ent, \In C1Long: \ent, \In C2Lat: \ent, \In C2Long: \ent}{\ent}{\\\indent\indent\indent
	\sqrt{(C1Lat-C2Lat)^{2}+(C1Long-C2Long)^{2}}
}


\subparagraph{}
\pred{esListaOrdenadaCreciente}{\In distancias: \TLista{\ent} }{\\\indent\indent\indent
	(\forall i:\ent)(0 \leq i < \longitud{distancias}-1 \implicaLuego  
	distancias[i] \leq distancias[i+1]\\
\indent\indent}

\subparagraph{}
\pred{noTieneValoresNulosNiNegativos}{\In distancias: \TLista{\ent} }{\\\indent\indent\indent
	(\forall i:\ent)(0 \leq i < \longitud{distancias} \implicaLuego  
	distancias[i] > 0\\
\indent\indent}

% ej 11
\subsection{Ejercicio 11}

\subparagraph{}
\pred {busquedaValida}{busqueda: seq$\langle (ItemIndividuo, dato) \rangle$}{\\\indent\indent\indent
	(\forall i : \ent)(0 \leq i < |busqueda| \implicaLuego atributoIndividuoEnRangoEsperado(busqueda[i]_0, busqueda[i]_1))\\\indent\indent
}

\subparagraph{}
\pred{interseccionVacia}{s: T, q: seq$\langle T \rangle$}{\\\indent\indent\indent
	(\forall i: \ent)(0 \leq i < |s| \implicaLuego ((\forall j: \ent)(0 \leq j < |q|) \implicaLuego s[i] \neq q[j]))\\\indent\indent
}

\subparagraph{}
\pred{cumpleBusqueda}{ind: individuo, busqueda seq$\langle(ItemIndividuo, dato)\rangle$}{\\\indent\indent\indent
	(\forall i: \ent)(0 \leq i < |busqueda| \implicaLuego ind[busqueda[i]_0] = busqueda[i]_1)\\\indent\indent
}

\subparagraph{}
\pred{contieneHogarDeIndividuo}{ind: individuo, th: seq$\langle hogar \rangle$}{\\\indent\indent\indent
	(\exists i: \ent)(0 \leq i < |th| \land_L ind[ord(INDCODUSU)]) = th[i][ord(HOGCODUSU)]\\\indent\indent
}



\section{Decisiones tomadas}

\subsection{Generales}
Se reemplaza el uso de los predicados `pertenece` y `estaContenido` por sus equivalentes matematicos $\in$ y $\subseteq$

\subsection{Ejercicio 1}
Para las coordenadas se decidió utilizar la convención de que la latitud se mide entre -90° y 90°, mientras que la longitud va desde -180° a 180°


\subsection{Ejercicio 4}
En este ejercicio se decidió que en el numerador, se pondría exclusivamente la cantidad de personas que viven en casa o departamento, que trabajan desde su casa y que a la vez tienen un ambiente dedicado para el trabajo, además de vivir en una gran ciudad, mientras que en el denominador, solo se consideran los individuos en hogares en una gran ciudad.

\subsection{Ejercicio 10}
Se agregó el chequeo de que las distancias contempladas no sean negativas, ya que no tendría sentido hablar de distancias negativas.

Al considerar las distancias, en cada iteración, se decidió incluir el limite inferior, pero no el superior, dejando afuera las casas que puedan estar a distancia distancias(|distancias| - 1) del centro

\end{document}
